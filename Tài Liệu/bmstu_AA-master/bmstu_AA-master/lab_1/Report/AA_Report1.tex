\documentclass[12pt]{report}
\usepackage[utf8]{inputenc}
\usepackage[russian]{babel}
%\usepackage[14pt]{extsizes}
\usepackage{listings}

% Для листинга кода:
\lstset{ %
language=python,                 % выбор языка для подсветки (здесь это С)
basicstyle=\small\sffamily, % размер и начертание шрифта для подсветки кода
numbers=left,               % где поставить нумерацию строк (слева\справа)
numberstyle=\tiny,           % размер шрифта для номеров строк
stepnumber=1,                   % размер шага между двумя номерами строк
numbersep=5pt,                % как далеко отстоят номера строк от подсвечиваемого кода
showspaces=false,            % показывать или нет пробелы специальными отступами
showstringspaces=false,      % показывать или нет пробелы в строках
showtabs=false,             % показывать или нет табуляцию в строках
frame=single,              % рисовать рамку вокруг кода
tabsize=2,                 % размер табуляции по умолчанию равен 2 пробелам
captionpos=t,              % позиция заголовка вверху [t] или внизу [b] 
breaklines=true,           % автоматически переносить строки (да\нет)
breakatwhitespace=false, % переносить строки только если есть пробел
escapeinside={\#*}{*)}   % если нужно добавить комментарии в коде
}

% Для измененных титулов глав:
\usepackage{titlesec, blindtext, color} % подключаем нужные пакеты
\definecolor{gray75}{gray}{0.75} % определяем цвет
\newcommand{\hsp}{\hspace{20pt}} % длина линии в 20pt
% titleformat определяет стиль
\titleformat{\chapter}[hang]{\Huge\bfseries}{\thechapter\hsp\textcolor{gray75}{|}\hsp}{0pt}{\Huge\bfseries}


% plot
\usepackage{pgfplots}
\usepackage{filecontents}
\usetikzlibrary{datavisualization}
\usetikzlibrary{datavisualization.formats.functions}
\begin{filecontents}{LevR.dat}
3 0.00006
4 0.00033
5 0.00033
6 0.00780
7 0.03876
8 0.20780
9 1.18171
\end{filecontents}

\begin{filecontents}{LevT.dat}
3 0.00003
4 0.00003
5 0.00005
6 0.00005
7 0.00007
8 0.00008
9 0.00009
\end{filecontents}

\begin{filecontents}{DamLevR.dat}
3 0.00006
4 0.00027
5 0.00143
6 0.00787
7 0.04130
8 0.23259
9 1.26665
\end{filecontents}

\begin{filecontents}{DamLevT.dat}
3 0.00003
4 0.00003
5 0.00005
6 0.00006
7 0.00007
8 0.00013
9 0.00012
\end{filecontents}


\begin{document}
%\def\chaptername{} % убирает "Глава"
\begin{titlepage}
	\centering
	{\scshape\LARGE МГТУ им. Баумана \par}
	\vspace{3cm}
	{\scshape\Large Лабораторная работа №1\par}
	\vspace{0.5cm}	
	{\scshape\Large По курсу: "Анализ алгоритмов"\par}
	\vspace{1.5cm}
	{\huge\bfseries Расстояние Левенштейна\par}
	\vspace{2cm}
	\Large Работу выполнила: Оберган Татьяна, ИУ7-55Б\par
	\vspace{0.5cm}
	\LargeПреподаватели:  Волкова Л.Л., Строганов Ю.В.\par

	\vfill
	\large \textit {Москва, 2019} \par
\end{titlepage}

\tableofcontents

\newpage
\chapter*{Введение}
\addcontentsline{toc}{chapter}{Введение}
\textbf{Расстояние Левенштейна} - минимальное количество операций вставки одного символа, удаления одного символа и замены одного символа на другой, необходимых для превращения одной строки в другую.

Расстояние Левенштейна применяется в теории информации и компьютерной лингвистике для:

\begin{itemize}
	\item исправления ошибок в слове
	\item сравнения текстовых файлов утилитой diff
	\item в биоинформатике для сравнения генов, хромосом и белков
\end{itemize}

Целью данной лабораторной работы является изучение метода динамического программирования на материале алгоритмов
Левенштейна и Дамерау-Левенштейна. 

Задачами данной лабораторной являются:
\begin{enumerate}
  	\item изучение алгоритмов Левенштейна и Дамерау-Левенштейна нахождения расстояния между строками;
	\item применение метода динамического программирования для матричной реализации указанных алгоритмов; 
	\item получение практических навыков реализации указанных алгоритмов: двух алгоритмов в матричной версии и одного из алгоритмов в рекурсивной версии; 
	\item сравнительный анализ линейной и рекурсивной реализаций выбранного алгоритма определения расстояния между строками по затрачиваемым ресурсам (времени и памяти); 
	\item экспериментальное подтверждение различий во временнóй эффективности рекурсивной и
нерекурсивной реализаций выбранного алгоритма определения расстояния между строками при
помощи разработанного программного обеспечения на материале замеров процессорного времени
выполнения реализации на варьирующихся длинах строк; 
	\item описание и обоснование полученных результатов в отчете о выполненной лабораторной
работе, выполненного как расчётно-пояснительная записка к работе. 
\end{enumerate}


\chapter{Аналитическая часть}
Задача по нахождению расстояния Левенштейна заключается в поиске минимального количества операций вставки/удаления/замены для превращения одной строки в другую.

При нахождении расстояния Дамерау — Левенштейна добавляется операция транспозиции (перестановки соседних символов).  
 
\textbf{Действия обозначаются так:} 
\begin{enumerate}
  	\item D (англ. delete) — удалить,
	\item I (англ. insert) — вставить,
	\item R (replace) — заменить,
	\item M(match) - совпадение.
\end{enumerate}

Пусть $S_{1}$ и $S_{2}$ — две строки (длиной M и N соответственно) над некоторым алфавитом, тогда расстояние Левенштейна можно подсчитать по следующей рекуррентной формуле:

\begin{displaymath}
D(i,j) = \left\{ \begin{array}{ll}
 0, & \textrm{$i = 0, j = 0$}\\
 i, & \textrm{$j = 0, i > 0$}\\
 j, & \textrm{$i = 0, j > 0$}\\
min(\\
D(i,j-1)+1,\\
D(i-1, j) +1, &\textrm{$j>0, i>0$}\\
D(i-1, j-1) + m(S_{1}[i], S_{2}[j])\\
),
  \end{array} \right.
\end{displaymath}

где $m(a,b)$ равна нулю, если $a=b$ и единице в противном случае; $min\{\,a,b,c\}$ возвращает наименьший из аргументов.

Расстояние Дамерау-Левенштейна вычисляется по следующей рекуррентной формуле:
\begin{displaymath}
D(i,j) = \left\{ \begin{array}{ll}
 0, & \textrm{$i = 0, j = 0$}\\
 i, & \textrm{$j = 0, i > 0$}\\
 j, & \textrm{$i = 0, j > 0$}\\
min(\\
D(i,j-1)+1,\\
D(i-1, j) +1, &\textrm{$j>0, i>0$}\\
D(i-1, j-1) + m(S_{1}[i], S_{2}[j])\\
D(i-2, j-2) + 1, &\textrm{if $i,j>1$ and $a_{i} = b_{j-1},a_{i-1}=b_{j} $}\\
)
  \end{array} \right.
\end{displaymath}



\chapter{Конструкторская часть}
\textbf{Требования к вводу:}
\begin{enumerate}
  	\item На вход подаются две строки
	\item uppercase и lowercase буквы считаются разными
\end{enumerate}
\textbf{Требования к программе:}
\begin{enumerate}
  	\item Две пустые строки - корректный ввод, программа не должна аварийно завершаться
\end{enumerate}


\chapter{Технологическая часть}
\section{Выбор ЯП}
В качестве языка программирования был выбран python т.к. я знакома с данным языком, имею представление о способах тестирования программы в рамках данного языка.

Время работы алгоритмов было замерено с помощью функции time() из библиотеки time.

\section{Сведения о модулях программы}
Программа состоит из:
\begin{itemize}
	\item main.py - главный файл программы, в котором располагаются алгоритмы и меню
	\item test.py - файл с тестами 
\end{itemize}

\begin{lstlisting}[label=some-code,caption=Функция нахождения расстояния Левенштейна рекурсивно]
def LevRecursion(str1, str2, output = False):
    if str1 ==  '' or str2 == '':
        return abs(len(str1) - len(str2))
    forfeit = 0 if (str1[-1] == str2[-1]) else 1
    return min(LevRecursion(str1, str2[:-1]) + 1,
               LevRecursion(str1[:-1], str2) + 1,
               LevRecursion(str1[:-1], str2[:-1]) + forfeit)
\end{lstlisting}

\begin{lstlisting}[label=some-code,caption=Функция нахождения расстояния Левенштейна матрично]
def LevTable(str1, str2, output = False):
    len_i = len(str1) + 1
    len_j = len(str2) + 1
    table = [[i + j for j in range(len_j)] for i in range(len_i)]
    
    for i in range(1, len_i):
        for j in range(1, len_j):
            forfeit = 0 if (str1[i - 1] == str2[j - 1]) else 1
            table[i][j] = min(table[i - 1][j] + 1,
                              table[i][j - 1] + 1,
                              table[i - 1][j - 1] + forfeit)
    if output:        
        OutputTable(table, str1, str2)
    return(table[-1][-1])
\end{lstlisting}

\begin{lstlisting}[label=some-code,caption=Функция нахождения расстояния Дамерау-Левенштейна рекурсивно]
def DamLevRecursion(str1, str2, output = False):
    if str1 ==  '' or str2 == '':
        return abs(len(str1) - len(str2))
    forfeit = 0 if (str1[-1] == str2[-1]) else 1
    res = min(DamLevRecursion(str1, str2[:-1]) + 1,
              DamLevRecursion(str1[:-1], str2) + 1,
              DamLevRecursion(str1[:-1], str2[:-1]) + forfeit)
    if (len(str1) >= 2 and len(str2) >= 2 and str1[-1] == str2[-2] and str1[-2] == str2[-1]):
        res = min(res, DamLevRecursion(str1[:-2], str2[:-2]) + 1)
    return res 
\end{lstlisting}

\begin{lstlisting}[label=some-code,caption=Функция нахождения расстояния Дамерау-Левенштейна матрично]
def DamLevTable(str1, str2, output = False):
    len_i = len(str1) + 1
    len_j = len(str2) + 1
    table = [[i + j for j in range(len_j)] for i in range(len_i)]
    
    for i in range(1, len_i):
        for j in range(1, len_j):
            forfeit = 0 if (str1[i - 1] == str2[j - 1]) else 1
            table[i][j] = min(table[i - 1][j] + 1,
                              table[i][j - 1] + 1,
                              table[i - 1][j - 1] + forfeit)
            if (i > 1 and j > 1) and str1[i-1] == str2[j-2] and str1[i-2] == str2[j-1]:
                table[i][j] = min(table[i][j], table[i-2][j-2] + 1)
    if output:        
        OutputTable(table, str1, str2)
    return(table[-1][-1])
\end{lstlisting}

\section{Тесты}
Тестирование было организовано с помощью библиотеки  \textbf{unittest}.
Было создано две вариации тестов:

В первой сравнивались результаты функции с реальным результатом.

Во второй сравнивались реузультаты двух функций(рекурсивной и табличной).
При сравнении результатов двух функций использовалась функция RandomString, которая генерирует случайную строку нужной длины.

\begin{lstlisting}[label=randStr,caption=Функция генерации случайной строки]
def RandomString(strLength = 5):
    letters = string.ascii_lowercase
    return ''.join(random.choice(letters) for i in range(strLength))
\end{lstlisting}


\chapter{Исследовательская часть}


Был проведен замер времени работы каждого из алгоритмов.

\begin{center}
	\begin{tabular}{|c c c c c|} 
 	\hline
	len & Lev(R) & DamLev(R) & Lev(T) & DamLev(T) \\ [0.5ex] 
 	\hline\hline
 	3 & 0.00006 & 0.00006 & 0.00003 & 0.00003\\
 	\hline
 	4 & 0.00033 & 0.00027 & 0.00003 & 0.00003\\
 	\hline
	5 & 0.00141 & 0.00143 & 0.00005 & 0.00005\\
	\hline
	6 & 0.00780 & 0.00787 & 0.00005 & 0.00006\\
	\hline
	7 & 0.03876 & 0.04130 & 0.00007 & 0.00007\\
	\hline
	8 & 0.20780 & 0.23259 & 0.00008 & 0.00013\\
	\hline
	9 & 1.18171 & 1.26665 & 0.00009 & 0.00012\\
	\hline
	\end{tabular}
\end{center}





\begin{tikzpicture}

\begin{axis}[
    	axis lines = left,
    	xlabel = $len$,
    	ylabel = {$time$},
	legend pos=north west,
	ymajorgrids=true
]
\addplot[color=red] table[x index=0, y index=1] {LevR.dat}; 
\addplot[color=orange] table[x index=0, y index=1] {DamLevR.dat};
\addplot[color=blue, mark=square] table[x index=0, y index=1] {LevT.dat};
\addplot[color=green, mark=square] table[x index=0, y index=1] {DamLevT.dat};

\addlegendentry{LevR}
\addlegendentry{DamLevR}
\addlegendentry{LevT}
\addlegendentry{DamLevT}
\end{axis}
\end{tikzpicture}


\par
Рекурсивные реализации сравнимы по времени между собой. При увеличении длины строк становится очевидна выигрышность по времени матричного варианта. Уже при длине в 9 символов матричная реализация в 10,000 раз быстрее.

\chapter*{Заключение}
\addcontentsline{toc}{chapter}{Заключение}
Был изучен метод динамического программирования на материале алгоритмов Левенштейна и Дамерау-Левенштейна.
Также изучены алгоритмы Левенштейна и Дамерау-Левенштейна нахождения расстояния между строками, получены практические навыки раелизации указанных алгоритмов
в матричной  и рекурсивных версиях. 

Экспериментально было подтверждено различие во временной эффективности рекурсивной и нерекурсивной реализаций выбранного алгоритма определения расстояния между строками при помощи разработаного программного обеспечения на материале замеров процессорного времени выполнения реализации на варьирующихся длинах строк. 

В результате исследований пришла к выводу, что матричная реализация данных алгоритмов заметно выигрывает по времени при росте длины строк.


\end{document}